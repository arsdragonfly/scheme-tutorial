\section{Lists}

\begin{frame}[fragile]
  \frametitle{Moving on to a higher order}
  \begin{itemize}
    \item What's the answer of \mintinline{scheme}{(1 2)}?
    \pause
    \item \emph{No answer. 1 is not a function that accepts 2 as an argument.}
    \pause
    \item What's the answer of \mintinline{scheme}{(foo 2)}?
    \pause
    \item \emph{No answer. \mintinline{scheme}{foo} is not a function yet.}
    \pause
    \item What's the answer of \mintinline{scheme}{(quote (1 2))}?
    \pause
    \item \emph{\mintinline{scheme}{(1 2)}}.
    \pause
    \item What's the answer of \mintinline{scheme}{'(1 2)}?
    \pause
    \item \emph{\mintinline{scheme}{(1 2)}. Same as before.}
  \end{itemize}
\end{frame}

\begin{frame}[fragile]
    \frametitle{Moving on to a higher order}
    \begin{itemize}
      \item What's the answer of \mintinline{scheme}{(atom? 1)}?
      \item \mintinline{scheme}{#t}.
      \pause
      \item What's the answer of \mintinline{scheme}{(atom? '())}?
      \item \mintinline{scheme}{#f}, \emph{because it is an empty list.}
      \pause
      \item What's the answer of \mintinline{scheme}{(atom? '(1 (2)))}?
      \item \mintinline{scheme}{#f}, \emph{because it is a list.}
      \pause
      \item What's the answer of \mintinline{scheme}{(null? '())}?
      \item \mintinline{scheme}{#f}, \emph{because it is an empty list.}
      \pause
      \item What's the answer of \mintinline{scheme}{(null? '(1 (2)))}?
      \item \mintinline{scheme}{#f}, \emph{because it is a non-empty list.}
    \end{itemize}
\end{frame}

\begin{frame}[fragile]
    \frametitle{Moving on to a higher order}
    \begin{itemize}
      \item What's the answer of \mintinline{scheme}{(car '())}?
      \item \emph{No answer, because it's an empty list.}
      \pause
      \item What's the answer of \mintinline{scheme}{(car '(1 (2)))}?
      \item \emph{1.}
      \pause
      \item What's the answer of \mintinline{scheme}{(cdr '())}?
      \item \emph{No answer, because it's an empty list.}
      \pause
      \item What's the answer of \mintinline{scheme}{(cdr '(1 (2)))}?
      \item \mintinline{scheme}{'((2))}. \emph{Note that it's different from } \mintinline{scheme}{'(2)} !
    \end{itemize}
\end{frame}

\begin{frame}[fragile]
  \frametitle{Moving on to a higher order}
  \begin{itemize}
    \item What's the answer of \mintinline{scheme}{(cons 1 '((2)))}?
    \item \emph{\mintinline{scheme}{'(1 (2))}}.
    \pause
    \item Can we build up the list from the empty list?
    \pause
    \item \emph{\mintinline{scheme}{(cons 1 (cons (cons 2 '()) '())))}}.
  \end{itemize}
\end{frame}

\begin{frame}[fragile]
  \frametitle{Moving on to a higher order}
  \begin{itemize}
    \item Define a function that removes all numbers equal to \mintinline{scheme}{x} in a list of atoms.
    \item \emph{Sure.}
  \end{itemize}
  \begin{minted}{scheme}
    (define multirember
      (lambda (x lat)
        (cond
          ((null? lat) '())
          ((= (car lat) x) (multirember x (cdr lat)))
          (else (cons (car lat) (multirember x (cdr lat)))))))
  \end{minted}
  \pause
  \begin{itemize}
    \item What are the steps to go through when dealing with a list of atoms?
    \item We always ask \mintinline{scheme}{(null? lat)} first, then ask other questions.
    \item What if we're dealing with list of lists?
    \pause
    \item We ask \mintinline{scheme}{(null? l)}, \mintinline{scheme}{(atom? (car l))} and other questions.
  \end{itemize}
\end{frame}

\begin{frame}[fragile]
  \frametitle{Moving on to a higher order}
  \begin{itemize}
    \item Could you define a function that removes all numbers less than \mintinline{scheme}{x}?
    \item \emph{Isn't this easy?}
  \end{itemize}
  \begin{minted}{scheme}
    (define multirember2
      (lambda (x lat)
        (cond
          ((null? lat) '())
          ((< (car lat) x) (multirember x (cdr lat)))
          (else (cons (car lat) (multirember x (cdr lat)))))))
  \end{minted}
\end{frame}

\begin{frame}[fragile]
  \frametitle{Moving on to a higher order}
  \begin{itemize}
    \item Could you define a function that removes all numbers greater than \mintinline{scheme}{x}?
    \item \emph{Isn't this also easy?}
  \end{itemize}
  \begin{minted}{scheme}
    (define multirember2
      (lambda (x lat)
        (cond
          ((null? lat) '())
          ((> (car lat) x) (multirember x (cdr lat)))
          (else (cons (car lat) (multirember x (cdr lat)))))))
  \end{minted}
\end{frame}

\begin{frame}[fragile]
  \frametitle{Moving on to a higher order}
  \begin{itemize}
    \item Is it really so easy as you said?
    \item \emph{What's so hard about this? we just copy the whole definition and change what we want.}
    \pause
    \item Is there a way to not copy the whole thing?
    \item \emph{What does it mean?}
    \pause
    \item Look at this function definition.
  \end{itemize}
  \begin{minted}{scheme}
(define mr-f
  (lambda (tester)
    (lambda (x lat)
      (cond
        ((null? lat) '())
        ((tester (car lat) x) ((mr-f tester) x (cdr lat)))
        (else (cons (car lat) ((mr-f tester) x (cdr lat))))))))
  \end{minted}
\end{frame}

\begin{frame}[fragile]
  \frametitle{Moving on to a higher order}
  \begin{itemize}
    \item What's the answer of \mintinline{scheme}{((mr-f =) 2 '(1 2 3))}?
    \pause
    \item \mintinline{scheme}{'(1 3)}.
    \pause
    \item What's the answer of \mintinline{scheme}{((mr-f <) 2 '(1 2 3))}?
    \pause
    \item \mintinline{scheme}{'(2 3)}.
    \pause
    \item What's the answer of \mintinline{scheme}{((mr-f >) 2 '(1 2 3))}?
    \pause
    \item \mintinline{scheme}{'(1 2)}.
    \pause
    \item What's so special about \mintinline{scheme}{mr-f}?
    \pause
    \item \emph{It accepts another function as an argument. Such functions are called \textcolor{RoyalBlue}{higher-order functions}.}
  \end{itemize}
\end{frame}

\begin{frame}[fragile]
  \frametitle{Add some curry}
  \begin{itemize}
    \item Let's write the plus function slightly differently.
    \item \emph{\mintinline{scheme}{(define o-plus (lambda (x) (lambda (y) (+ x y))))}}.
    \item Can we write all functions with multiple arguments in this way?
    \pause
    \item \emph{Sure. Why not? But What's the difference?}
    \item Suppose we have the following function:
    \begin{minted}{scheme}
(define maaaap
  (lambda (f)
    (lambda (lat)
      (cond
        ((null? lat) '())
        (else (cons (f (car lat))
                    ((maaaap f) (cdr lat))))))))
    \end{minted}
    \item How to add 2 to every element of \mintinline{scheme}{'(1 2 3)}?
  \end{itemize}
\end{frame}

\begin{frame}[fragile]
  \frametitle{Add some curry}
    \begin{minted}{scheme}
(define maaaap
  (lambda (f)
    (lambda (lat)
      (cond
        ((null? lat) '())
        (else (cons (f (car lat))
                    ((maaaap f) (cdr lat))))))))
    \end{minted}
  \begin{itemize}
    \item How about this: \mintinline{scheme}{((maaaap (oplus 1)) '(1 2 3))}?
    \pause
    \item \emph{Yes, it works. Can we multiply it by 3?}
    \item Sure. This is left for you as an exercise.
    \pause
    \item Does the process of taking functions apart into many lambdas get a name?
    \item \emph{Yes. It's named \textcolor{RoyalBlue}{Currying} in honor of \textcolor{Blue}{Haskell B. Curry}.}
  \end{itemize}
\end{frame}

\begin{frame}[fragile]
  \frametitle{Are they really the same?}
  \begin{itemize}
    \item We put the definition of \mintinline{scheme}{o=} here for convenience.
    \begin{minted}{scheme}
      (define o=
        (lambda (x y)
          (cond
            ((zero? x) (zero? y))
            ((zero? y) #f))
            (else (o= (sub1 x) (sub1 y)))))
    \end{minted}
    Now, try to write the Fibonacci function.
  \end{itemize}
\end{frame}

\begin{frame}[fragile]
  \frametitle{Are they really the same?}
  \begin{minted}{scheme}
    (define fibonacci
      (lambda (x)
        (cond
          ((zero? x) 1)
          ((zero? (sub1 x) 1))
          (else ...))))
  \end{minted}
  \begin{itemize}
    \item The code above is a good place to start.
    \pause
    \item \emph{Doesn't the following definition look natural?}
    \begin{minted}{scheme}
      (define fib
        (lambda (x)
          (cond
            ((zero? x) 1)
            ((zero? (sub1 x)) 1)
            (else (+ (fib (sub1 x)) (fib (- x 2)))))))
    \end{minted}
  \end{itemize}
\end{frame}

\begin{frame}[fragile]
  \frametitle{Are they really the same?}
  \begin{itemize}
    \item It might look natural, but it's definitely not the optimal.
    \item \emph{What does it mean?}
    \pause
    \item See how long it takes for \mintinline{scheme}{(fib 10000)} to work out.
  \end{itemize}
\end{frame}

\begin{frame}[fragile]
  \frametitle{Are they really the same?}
  \begin{itemize}
    \item How about this definition:
  \end{itemize}
  \begin{minted}{scheme}
  (define fibonacci-helper
    (lambda (r prev pprev)
      (cond
        ((zero? r) (+ prev pprev))
        (else (fibonacci-helper (sub1 r) (+ prev pprev) prev)))))
  (define fibonacci-alt
    (lambda (x)
      (fibonacci-helper x 0 1)))
  \end{minted}
  \begin{itemize}
    \pause
    \item Do we need to be careful about how fast our program runs?
    \pause
    \item \emph{Absolutely.}
  \end{itemize}
\end{frame}
